% Compile with XeLaTeX
\documentclass{amsart}

\usepackage{fontenc}
\usepackage{xunicode}
\usepackage{bbold}
\usepackage[usenames,dvipsnames]{color}
\usepackage{url}

\usepackage{tikz}

\newcommand\mvj[1]{\textcolor{CadetBlue}{\textbf{MVJ: }#1}}
\newcommand\jh[1]{\textcolor{ForestGreen}{\textbf{JH: }#1}}
\newcommand\ps[1]{\textcolor{Violet}{\textbf{PS: }#1}}
\newcommand\de[1]{\textcolor{RoyalPurple}{\textbf{DE: }#1}}

\begin{document}

\title{Point cloud topology on algebraic varieties}
\author{Mikael Vejdemo-Johansson \and %
  Jon Hauenstein \and %
  Primoz Skraba \and %
  David Eklund}

\maketitle

\begin{abstract}
  We propose a method to compute betti numbers from algebraic varieties by numerical sampling and subsequent homology estimation with persistent homology techniques.
\end{abstract}

\section{Introduction}
\label{sec:introduction}

Persistent homology is cool. \cite{elz2000,c09TandD}


\section{Constructing a point cloud}
\label{sec:constr-point-cloud}

\section{Analyzing the point cloud}
\label{sec:analyz-point-cloud}


\section{Experiments}
\label{sec:experiments}

We computed using \texttt{jPlex}. \cite{jplex08}

\section{Conclusions}
\label{sec:conclusions}




\bibliographystyle{apalike}
\bibliography{realpoints}

\end{document}

%%% Local Variables: 
%%% mode: latex
%%% TeX-master: t
%%% End: 
